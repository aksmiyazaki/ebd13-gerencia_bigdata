%%%%%%%%%%%%%%%%%%%%%%%%%%%%%%%%%%%%%%%%%
% Lachaise Assignment
% LaTeX Template
% Version 1.0 (26/6/2018)
%
% This template originates from:
% http://www.LaTeXTemplates.com
%
% Authors:
% Marion Lachaise & François Févotte
% Vel (vel@LaTeXTemplates.com)
%
% License:
% CC BY-NC-SA 3.0 (http://creativecommons.org/licenses/by-nc-sa/3.0/)
% 
%%%%%%%%%%%%%%%%%%%%%%%%%%%%%%%%%%%%%%%%%

%----------------------------------------------------------------------------------------
%	PACKAGES AND OTHER DOCUMENT CONFIGURATIONS
%----------------------------------------------------------------------------------------

\documentclass{article}

\usepackage{graphicx}
\input{structure.tex} % Include the file specifying the document structure and custom commands

%----------------------------------------------------------------------------------------
%	ASSIGNMENT INFORMATION
%----------------------------------------------------------------------------------------

\title{EBD13 Gerência de Dados e Computação em Nuvem} % Title of the assignment

\author{Alexandre Miyazaki\\ \texttt{aksmiyazaki@gmail.com}} % Author name and email address

\date{Universidade Federal do Rio Grande do Sul} % University, school and/or department name(s) and a date

%----------------------------------------------------------------------------------------

\begin{document}

\maketitle % Print the title

\section*{Objetivo do Trabalho} % Unnumbered section

O objetivo do trabalho era:

\begin{itemize}
\item Criar uma infraestrutura computacional em nuvem;
\item Realizar o deploy de um ambiente que seja capaz de analisar dados nesse ambiente;
\item Realizar uma análise simples.
\end{itemize}

Nas próximas seções estão os passos para atingir tais objetivos.


\section{Criando Infraestrutura} % Numbered section

Para criação da infraestrutura, foi escolhida a AWS, pois foi o ambiente no qual o professor realizou o laboratório.

\begin{figure}[h]
  \includegraphics[width=\linewidth]{img/machines_created.png}
  \caption{Máquinas criadas na \emph{cloud}.}
  \label{fig:fig_maq_criad}
\end{figure}

A Figura \ref{fig:fig_maq_criad} mostra a infraestrutura de máquinas. Foram criados 4 nós na AWS, sendo um deles o Mestre e os demais, escravos. 
A parte de segurança da rede na própria AWS foi negligenciada, deixando todas as portas abertas. Abaixo, a Figura \ref{fig:machine_config} mostra parte da configuração das máquinas na AWS.


\begin{figure}[h]
  \includegraphics[width=\linewidth]{img/machine_config.png}
  \caption{Parte da configuração dos nós.}
  \label{fig:machine_config}
\end{figure}

Antes de passarmos para a próxima Seção, precisamos conseguir acessar as máquinas criadas. O trabalho foi realizado
em uma máquina Linux, de forma que isso se torna fácil via ssh. A Figura \ref{fig:machine_access} mostra o processo para acessar a máquina via terminal.

\begin{figure}[h]
  \includegraphics[width=\linewidth]{img/machine_access.png}
  \caption{Acesso aos nós via ssh..}
  \label{fig:machine_access}
\end{figure}



\section{Deploy de Ambiente}
\end{document}